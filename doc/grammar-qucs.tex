%
% This document contains the qucs netlist grammar
%
% Copyright (C) 2005 Raimund Jacob <raimi@lkcc.org>
%
% Permission is granted to copy, distribute and/or modify this document
% under the terms of the GNU Free Documentation License, Version 1.1
% or any later version published by the Free Software Foundation.
%

\chapter{Qucs netlist grammar}
\label{sec:qucsgrammar}

%
% Grammar environment based on Martin Grabmueller's original one
%
\newcommand\tok[1]{`{\bf #1}'}  % String-f"ormiges Token
\newcommand\etok[1]{{\bf #1}}   % Token, das sp"ater erkl"art wird
\newcommand\stok[1]{{\em #1}}   % Spezieller Wert
\newcommand\ntok[1]{{\bf #1}}   % Token ohne Anf"uhrungszeichen
\newcommand\bsl[1]{\tok{$\backslash$#1}}
\newenvironment{grammar}%
  {\newcommand\produces[2]{##1 \> $\rightarrow$ \> ##2 \\}
   \newcommand\orproduces[1]{\> \> \makebox[0pt][r]{$|$ }##1 \\}
   \newcommand\opt[1]{[ ##1 ]}
   \newcommand\rep[1]{\{ ##1 \}}
   \newcommand\alt[0]{$|$}
   \newcommand\group[1]{( ##1 )}
   \newcommand\rutsch[0]{\\\> \>}
   \newcommand\emptyprod{{$\varepsilon$}}
   \newcommand\heading[1]{\rule{\linewidth}{1pt} \\{\bf ##1}\\[2ex]}
   \newcommand\separator{\rule{\linewidth}{1pt} \\}
   \begin{tabbing}
   \qquad\qquad\qquad\qquad\qquad \= \qquad \= \kill}
  {\end{tabbing}}

%
% How the grammar works
%
The grammer is presented using the Extended Backus-Naur Form (EBNF)
which works as follows:

\vskip0.5em
\noindent
\begin{tabular}{ll}
$A \rightarrow B$&
Nonterminal $A$ produces form $B$.\\

$B | C$&
Produces $B$ or $C$.\\

$\{ A \}$&
Arbitrary repitition of form $A$. No repitition is allowed as well.\\

[ $A$ ]&
From $A$ is optional.\\

$(A)$&
Grouping, stands for $A$ itself.\\

$\varepsilon$& The empty form.\\
\end{tabular}
\vskip0.5em

\noindent
Nonterminals are set in normal font, terminals are in bold font.

%
% Syntax
%
\begin{grammar}
\heading{Syntactic Structure}

\produces{Input}{\rep{InputLine}}

\produces{InputLine}{SubcircuitBody}
\orproduces{EquationLine}
\orproduces{ActionLine}
\orproduces{DefinitionLine}

\produces{ActionLine}{\tok{.} \ntok{Identifier} \tok{:} PairList \ntok{Eol}}

\produces{DefinitionLine}{\ntok{Identifier} \tok{:} \ntok{Identifer}
         \rep{\ntok{Identifier}} PairList \ntok{Eol}}

\produces{PairList}{\rep{\ntok{Identifier} \tok{=} Value}}

\produces{Value}{PropertyValue}
\orproduces{\tok{``} PropertyValue \tok{``}}

\produces{PropertyValue}{\ntok{Identifier}}
\orproduces{PropertyReal}
\orproduces{\tok{[} PropertyReal \opt{\rep{\tok{;} PropertyReal}} \tok{]}}

\produces{PropertyReal}{\ntok{Real} \opt{\ntok{Scale} \opt{\ntok{Unit}}}}

\produces{EquationLine}{\tok{Eqn} \tok{:}
   \ntok{Identifier} Equation \rep{Equation}}
\produces{Equation}{\ntok{Identifier} \tok{=} \tok{``} Expression \tok{``}}

\produces{Expression}{Constant}
\orproduces{Reference}
\orproduces{Application}
\orproduces{\tok{(} Expression \tok{)}}

\produces{Constant}{\ntok{Real}}
\orproduces{\ntok{Imag}}
\orproduces{\ntok{Character}}
\orproduces{\ntok{String}}
\orproduces{Range}

\produces{Range}{\opt{\ntok{Real}} \tok{:} \opt{\ntok{Real}}}

\produces{Reference}{\ntok{Identifier}}

\produces{Application}{\ntok{Identifier}
   \tok{(} \ntok{Expression} \opt{\rep{\tok{,} \ntok{Expression} }} \tok{)}}
\orproduces{\ntok{Reference}
   \tok{[} \ntok{Expression} \opt{\rep{\tok{,} \ntok{Expression} }} \tok{]}}
\orproduces{\group{\tok{+} \alt{} \tok{-}} Expression}
\orproduces{Expression
   \group{\tok{+} \alt \tok{-} \alt \tok{*} \alt \tok{/} \alt
          \tok{\%} \alt \tok{\^{}}}
   Expression}

\produces{SubcircuitBody}{DefBegin \rep{DefBodyLine} DefEnd}

\produces{DefBegin}{\tok{.} \tok{Def} \tok{:} \ntok{Identifier}
   \rep{\ntok{Identifier}} \ntok{Eol}}

\produces{DefBodyLine}{DefinitionLine}
\orproduces{SubcircuitBody}
\orproduces{\ntok{Eol}}

\produces{DefEnd}{\tok{.} \tok{Def} \tok{:} \tok{End}}

\end{grammar}

%
% Lexical stuff
%
\begin{grammar}
\heading{Lexical structure}
\produces{Identifier}{TBI}
\produces{Real}{TBI}
\produces{Imag}{TBI}
\produces{Character}{TBI}
\produces{String}{TBI}
\produces{Scale}{TBI}
\produces{Unit}{TBI}
\end{grammar}

