\documentclass[10pt]{report}
\usepackage{a4wide}
\usepackage{epsfig}
\usepackage{array}
\usepackage{amsmath}
\usepackage{SIunits}
\usepackage{psfrag}
\usepackage{relsize}
\usepackage[section]{placeins}
\usepackage{listings}

\newif\ifpdf
\ifx\pdfoutput\undefined
  \pdffalse
\else
  \pdfoutput=1
  \pdftrue
\fi

\ifpdf
\pdfcompresslevel=9
\pdfinfo {
  /Title   (Qucs)
  /Subject (Technical Papers)
  /Author  (Stefan Jahn)
}
\fi

\makeatletter
\def\thickhrulefill{\leavevmode \leaders \hrule height 1pt\hfill \kern \z@}
\renewcommand{\maketitle}{\begin{titlepage}%
    \let\footnotesize\small
    \let\footnoterule\relax
    \parindent \z@
    \reset@font
    \null\vfil
    \vspace*{3cm}
    \begin{flushleft}
      \bf \huge \@title
    \end{flushleft}
    \par
    \hrule height 3pt
    \par
    \begin{flushright}
      \LARGE Technical Papers \par
    \end{flushright}
    \vskip 60\p@
    \vfill


    \begin{flushright}
      \Large \@author \par
    \end{flushright}

    \hrule height 3pt \par

\vspace*{24pt}

Copyright \copyright{} 2003, 2004 Michael Margraf 
\textless margraf@mwt.ee.tu-berlin.de\textgreater \par
Copyright \copyright{} 2003, 2004 Stefan Jahn 
\textless jahn@mwt.ee.tu-berlin.de\textgreater \par

\vspace*{12pt}

Permission is granted to copy, distribute and/or modify this document
under the terms of the GNU Free Documentation License, Version 1.1 or
any later version published by the Free Software Foundation.  A copy
of the license is included in the section entitled "GNU Free
Documentation License".

\vspace*{1cm}

  \end{titlepage}%
  \setcounter{footnote}{0}%
}
\makeatother

\author{Michael Margraf \\ Stefan Jahn}
\title{Qucs}
\date{}

\begin{document}

\maketitle

\tableofcontents

\setlength{\parindent}{0pt}
\newpage

\chapter*{Mathematical expressions and conversions}
\addcontentsline{toc}{chapter}{Mathematical expressions and conversions}

\section*{Scattering parameters}
\addcontentsline{toc}{section}{Scattering parameters}

\subsection*{Recalculating $\mathbf{50\ohm}$-S-parameters for arbitrary port impedances}
\addcontentsline{toc}{subsection}{Recalculating $50\ohm$-S-parameters for arbitrary port impedances}

During S-parameter simulation it is necessary to have all components
in a circuit normalized to the same impedance.  In the field of high
frequency techniques this is usually $50\ohm$.  In order to allow port
impedances other than $50\ohm$ in a simulation the following two step
process must be applied to the resulting S-parameter analysis.

\begin{equation}
\left[\underline{N}\right] = 
\left(\left[\underline{S}\right] - \left[\underline{R}\right]\right) \cdot
\left(\left[\underline{E}\right] - \left[\underline{R}\right] \cdot \left[\underline{S}\right]\right)^{-1}
\end{equation}

\begin{equation}
\underline{N}_{nm} = \underline{S}_{nm}\cdot \sqrt{\dfrac{Z^{m}}{Z^{n}}}\cdot
\dfrac{Z^{n} + Z_{0}}{Z^{m} + Z_{0}}
\end{equation}

With\\

\begin{tabular}{rll}
$Z_{0}$ & = & $50\ohm$\\& &\\
$\left[\underline{E}\right]$ & = &
$\begin{pmatrix}
1 & 0 & \ldots & 0\\
0 & 1 & \ldots & 0\\
\vdots & \vdots & \ddots & \vdots\\
0 & 0 & \ldots & 1\\
\end{pmatrix}$
identity matrix\\& &\\
$\left[\underline{S}\right]$ & = & original $50\ohm$-S-parameter matrix\\& &\\
$\left[\underline{N}\right]$ & = & recalculated scattering matrix\\& &\\
$\left[\underline{R}\right]$ & = &
$\begin{pmatrix}
\underline{r}(Z_{1}) & 0 & \ldots & 0\\
0 & \underline{r}(Z_{2}) & \ldots & 0\\
\vdots & \vdots & \ddots & \vdots\\
0 & 0 & \ldots & \underline{r}(Z_{n})\\
\end{pmatrix}$
reflection coefficient matrix\\& &\\
$\underline{r}(Z_{n})$ & = &
$\dfrac{Z_{n} - Z_{0}}{Z_{n} + Z_{0}}$
reflection coefficient of impedance at port n\\& &\\
\end{tabular}

And furthermore\\

\begin{tabular}{rll}
$\left[\underline{X}\right]^{-1}$ & = & 
inverted matrix of $\left[\underline{X}\right]$\\& &\\
$\underline{X}_{nm}$ & = & 
element of matrix $\left[\underline{X}\right]$ at row n and column m\\& &\\
\end{tabular}

\subsection*{Differential S-parameter ports}
\addcontentsline{toc}{subsection}{Differential S-parameter ports}

The implemented algorithm for the S-parameter analysis calculates
S-parameters in terms of the ground node.  In order to allow
differential S-parameters as well it is necessary to insert an ideal
impedance transformer with a turns ratio of 1:1 between the
differential port and the device under test.

\begin{figure}[ht]
\begin{center}
\includegraphics[width=12cm]{differential}
\end{center}
\caption{transformation of differential port into single ended port}
\label{fig:differential}
\end{figure}
\FloatBarrier

The S-parameter matrix of the inserted ideal transformer being a three
port device can be written as follows.

\begin{equation}
\begin{pmatrix}
S
\end{pmatrix}
= \dfrac{1}{3}\cdot
\begin{pmatrix}
1 & 2 & -2\\
2 & 1 & 2\\
-2 & 2 & 1\\
\end{pmatrix}
\end{equation}

This transformation can be applied to each S-parameter port in a
circuit regardless whether it is actually differential or not.\\

It is also possible to do the impedance transformation within this step
(for S-parameter ports with impedances different than $50\ohm$). This can
be done by using a transformer with an impedance ration of

\begin{equation}
r=T^2=\frac{50\ohm}{Z}
\end{equation}

With $Z$ being the S-parameter port impedance. The S-parameter matrix of
the inserted ideal transformer now writes as follows.

\begin{equation}
\begin{pmatrix}
S
\end{pmatrix}
= \dfrac{1}{2\cdot Z_0+Z}\cdot
\begin{pmatrix}
2\cdot Z_0-Z              & 2\cdot\sqrt{Z_0\cdot Z}  & -2\cdot\sqrt{Z_0\cdot Z}\\
2\cdot\sqrt{Z_0\cdot Z}   & Z                        & 2\cdot Z_0\\
-2\cdot\sqrt{Z_0\cdot Z}  & 2\cdot Z_0               & Z\\
\end{pmatrix}
\end{equation}

With $Z$ being the new S-parameter port impedance and $Z_0$ being $50\ohm$.

\section*{Scattering parameters of components}
\addcontentsline{toc}{section}{Scattering parameters of components}

\subsection*{Resistor}
\addcontentsline{toc}{subsection}{Resistor}

The scattering parameters of an ideal, ohmic resistor with resistance
$R$ writes as follows.

\begin{equation}
S_{11} = S_{22} = \frac{R}{2\cdot Z_0+R} \\
\end{equation}
\begin{equation}
S_{12} = S_{21} = 1-S_{11} = \frac{2\cdot Z_0}{2\cdot Z_0+R}
\end{equation}

\subsection*{Capacitor}
\addcontentsline{toc}{subsection}{Capacitor}

The scattering parameters of an ideal capacitor with capacitance $C$
writes as follows.

\begin{equation}
S_{11} = S_{22} = \frac{1}{2\cdot Z_0\cdot j\omega C+1} \\
\end{equation}
\begin{equation}
S_{12} = S_{21} = 1-S_{11}
\end{equation}

\subsection*{Inductor}
\addcontentsline{toc}{subsection}{Inductor}

The scattering parameters of an ideal inductor with inductance $L$
writes as follows.

\begin{equation}
S_{11} = S_{22} = \frac{j\omega L}{2\cdot Z_0 + j\omega L} \\
\end{equation}
\begin{equation}
S_{12} = S_{21} = 1-S_{11}
\end{equation}

\subsection*{DC Block}
\addcontentsline{toc}{subsection}{DC Block}

A DC block is a capacitor with an infinite capacitance.  The
scattering parameters, therefore, writes as follows.

\begin{equation}
\begin{pmatrix}
S
\end{pmatrix}
=
\begin{pmatrix}
0 & 1\\
1 & 0\\
\end{pmatrix}
\end{equation}

\subsection*{DC Feed}
\addcontentsline{toc}{subsection}{DC Feed}

A DC feed is an inductor with an infinite inductance.  The scattering
parameters, therefore, writes as follows.

\begin{equation}
\begin{pmatrix}
S
\end{pmatrix}
=
\begin{pmatrix}
1 & 0\\
0 & 1\\
\end{pmatrix}
\end{equation}

\subsection*{Bias T}
\addcontentsline{toc}{subsection}{Bias T}

A bias T is a combination of a DC block and a DC feed
(fig. \ref{fig:biast}).  The scattering parameters, therefore, writes
as follows.

\begin{equation}
\begin{pmatrix}
S
\end{pmatrix}
=
\begin{pmatrix}
0 & 1 & 0\\
1 & 0 & 0\\
0 & 0 & 1\\
\end{pmatrix}
\end{equation}

\begin{figure}[ht]
\begin{center}
\includegraphics[width=3.5cm]{biast}
\end{center}
\caption{bias t}
\label{fig:biast}
\end{figure}
\FloatBarrier

\subsection*{Transformer}
\addcontentsline{toc}{subsection}{Transformer}

Using the port numbers depicted in fig. \ref{fig:trafo}, the
scattering parameters of an ideal transformer with voltage
transformation ratio $T$ (number of turns) writes as follows.

\begin{equation}
S_{14} = S_{22} = S_{33} = S_{41} = \frac{1}{T^2+1}
\end{equation}
\begin{equation}
S_{12} = -S_{13} = S_{21} = -S_{24} = -S_{31} = S_{34} = -S_{42} = S_{43} = T\cdot S_{22}
\end{equation}
\begin{equation}
S_{11} = S_{23} = S_{32} = S_{44} = T\cdot S_{12}
\end{equation}

\begin{figure}[ht]
\begin{center}
\includegraphics[width=4cm]{trafo}
\end{center}
\caption{transformer}
\label{fig:trafo}
\end{figure}
\FloatBarrier

\subsection*{Symmetrical transformer}
\addcontentsline{toc}{subsection}{Symmetrical transformer}

Using the port numbers depicted in fig. \ref{fig:symtrafo}, the
scattering parameters of an ideal, symmetrical transformer with
voltage transformation ratio (number of turns) $T_1$ and $T_2$,
respectively, writes as follows.

\begin{equation}
denom = T_1^2+T_2^2+T_1^2\cdot T_2^2
\end{equation}
\begin{eqnarray}
S_{11} = S_{66} = \frac{T_2^2}{denom}  &  \qquad S_{16} = S_{61} = 1-S_{11} \\
S_{44} = S_{55} = \frac{T_1^2}{denom}  &  \qquad S_{45} = S_{54} = 1-S_{44} \\
S_{22} = S_{33} = \frac{T_1^2\cdot T_2^2}{denom}  &  \qquad S_{23} = S_{32} = 1-S_{22} \\
\end{eqnarray}
\begin{equation}
S_{12} = S_{21} = -S_{13} = -S_{31} = -S_{26} = -S_{62} = S_{36} = S_{63}
       = \frac{T_1\cdot T_2^2}{denom}
\end{equation}
\begin{equation}
-S_{24} = -S_{42} = S_{25} = S_{52} = S_{34} = S_{43} = -S_{35} = -S_{53}
       = \frac{T_1^2\cdot T_2}{denom}
\end{equation}
\begin{equation}
-S_{14} = -S_{41} = S_{15} = S_{51} = S_{46} = S_{64} = -S_{56} = -S_{65}
       = \frac{T_1\cdot T_2}{denom}
\end{equation}

\begin{figure}[ht]
\begin{center}
\includegraphics[width=4cm]{symtrafo}
\end{center}
\caption{symmetrical transformer}
\label{fig:symtrafo}
\end{figure}
\FloatBarrier

\subsection*{Attenuator}
\addcontentsline{toc}{subsection}{Attenuator}

The scattering parameters of an ideal attenuator with attenuation $L$
(loss) in reference to the impedance $Z_{ref}$ writes as follows.

\begin{equation}
r = \frac{Z_0-Z_{ref}}{Z_0+Z_{ref}}
\end{equation}
\begin{equation}
S_{11} = S_{22} = \frac{r\cdot(1-L^2)}{L^2-r^2}
\end{equation}
\begin{equation}
S_{12} = S_{21} = \frac{L\cdot(1-r^2)}{L^2-r^2}
\end{equation}

\subsection*{Isolator}
\addcontentsline{toc}{subsection}{Isolator}

An isolator is a one-way two-port, transporting incoming waves
lossless from the input (port 1) to the output (port 2), but consuming
all waves flowing into the output. With the reference impedance of the
input $Z_1$ and the one of the output $Z_2$, the scattering parameters
of an ideal isolator writes as follows.

\begin{equation}
S_{11} = \frac{Z_1-Z_0}{Z_1+Z_0}
\end{equation}
\begin{equation}
S_{12} = 0
\end{equation}
\begin{equation}
S_{22} = \frac{Z_2-Z_0}{Z_2+Z_0}
\end{equation}
\begin{equation}
S_{21} = \sqrt{1-(S_{11})^2}\cdot\sqrt{1-(S_{22})^2}
\end{equation}

\subsection*{Circulator}
\addcontentsline{toc}{subsection}{Circulator}

A circulator is a 3-port device, transporting incoming waves lossless
from port 1 to port 2, from port 2 to port 3 and from port 3 to port
1.  In all other directions, there is no energy flow.  With the
reference impedances $Z_1$, $Z_2$ and $Z_3$ for the ports 1, 2 and 3
the scattering matrix of an ideal circulator writes as follows.

\begin{equation}
denom = 1-r_1\cdot r_2\cdot r_3
\end{equation}
\begin{equation}
r_1 = \frac{Z_0-Z_1}{Z_0+Z_1} \qquad,\qquad 
r_2 = \frac{Z_0-Z_2}{Z_0+Z_2} \qquad,\qquad 
r_3 = \frac{Z_0-Z_3}{Z_0+Z_3}
\end{equation}
\begin{equation}
S_{11} = \frac{r_2\cdot r_3 - r_1}{denom} \qquad,\qquad 
S_{22} = \frac{r_1\cdot r_3 - r_2}{denom} \qquad,\qquad 
S_{33} = \frac{r_1\cdot r_2 - r_3}{denom}
\end{equation}
\begin{equation}
S_{12} = \sqrt{\frac{Z_2}{Z_1}}\cdot\frac{Z_1+Z_0}{Z_2+Z_0}\cdot\frac{r_3\cdot(1-r_1^2)}{denom}
\qquad,\qquad 
S_{13} = \sqrt{\frac{Z_3}{Z_1}}\cdot\frac{Z_1+Z_0}{Z_3+Z_0}\cdot\frac{1-r_1^2}{denom}
\end{equation}
\begin{equation}
S_{21} = \sqrt{\frac{Z_1}{Z_2}}\cdot\frac{Z_2+Z_0}{Z_1+Z_0}\cdot\frac{1-r_2^2}{denom}
\qquad,\qquad 
S_{23} = \sqrt{\frac{Z_3}{Z_2}}\cdot\frac{Z_2+Z_0}{Z_3+Z_0}\cdot\frac{r_1\cdot(1-r_2^2)}{denom}
\end{equation}
\begin{equation}
S_{31} = \sqrt{\frac{Z_1}{Z_3}}\cdot\frac{Z_3+Z_0}{Z_1+Z_0}\cdot\frac{r_2\cdot(1-r_3^2)}{denom}
\qquad,\qquad 
S_{32} = \sqrt{\frac{Z_2}{Z_3}}\cdot\frac{Z_3+Z_0}{Z_2+Z_0}\cdot\frac{1-r_3^2}{denom}
\end{equation}

\subsection*{Phase shifter}
\addcontentsline{toc}{subsection}{Phase shifter}

The scattering parameters of an ideal phase shifter with phase shift
$\phi$ and reference impedance $Z_{ref}$ writes as follows.

\begin{equation}
r = \frac{Z_0-Z_{ref}}{Z_0+Z_{ref}}
\end{equation}
\begin{equation}
S_{11} = S_{22} = \frac{r\cdot\left(\exp(j\cdot 2\phi)-1\right)}{1-r^2\cdot\exp(j\cdot 2\phi)}
\end{equation}
\begin{equation}
S_{12} = S_{21} = \frac{(1-r^2)\cdot\exp(j\cdot\phi)}{1-r^2\cdot\exp(j\cdot 2\phi)}
\end{equation}


\subsection*{Gyrator}
\addcontentsline{toc}{subsection}{Gyrator}

A gyrator is an impedance inverter.  Thus, for example, it converts a
capacitance into an inductance and vice versa.  The scattering matrix
of an ideal gyrator with the ratio $R$ writes as follows.

\begin{equation}
r = \frac{R}{Z_0} = \frac{1}{G\cdot Z_0}
\end{equation}
\begin{equation}
S_{11} = S_{22} = S_{33} = S_{44} = \frac{R^2}{4\cdot Z_0^2 + R^2} = \frac{r^2}{r^2+4}
\end{equation}
\begin{equation}
S_{14} = S_{23} = S_{32} = S_{41} = 1-S_{11}
\end{equation}
\begin{equation}
S_{12} = -S_{13} = -S_{21} = S_{24} = S_{31} = -S_{34} = -S_{42} = S_{43} = \frac{2\cdot r}{r^2+4}
\end{equation}

\subsection*{Voltage and current sources}
\addcontentsline{toc}{subsection}{Voltage and current sources}

All voltage sources (AC and DC) are short circuits and therefore their
S-parameter matrix equals the one of the DC block.  All current
sources are open circuits and therefore their S-parameter matrix
equals the one of the DC feed.

\subsection*{Controlled voltage and current sources}
\addcontentsline{toc}{subsection}{Controlled voltage and current sources}

The scattering matrix of the voltage controlled current source
depicted in fig. \ref{fig:xCCS} (left) writes as follows ($\tau$ is
time delay).

\begin{equation}
S_{11} = S_{22} = S_{33} = S_{44} = 1
\end{equation}
\begin{equation}
S_{12} = S_{13} = S_{14} = S_{23} = S_{32} = S_{41} = S_{42} = S_{43} = 0
\end{equation}
\begin{equation}
S_{21} = S_{34} = 2\cdot G\cdot \exp(j\pi-j\omega\tau)
\end{equation}
\begin{equation}
S_{24} = S_{31} = 2\cdot G\cdot \exp(-j\omega\tau)
\end{equation}

\begin{figure}[ht]
\begin{center}
\includegraphics[width=8cm]{xCCS}
\end{center}
\caption{voltage controlled current source (left) and current controlled current source (right)}
\label{fig:xCCS}
\end{figure}
\FloatBarrier

The scattering matrix of the current controlled current source
depicted in fig. \ref{fig:xCCS} (right) writes as follows ($\tau$ is
time delay).

\begin{equation}
S_{14} = S_{22} = S_{33} = S_{41} = 1
\end{equation}
\begin{equation}
S_{11} = S_{12} = S_{13} = S_{23} = S_{32} = S_{42} = S_{43} = S_{44} = 0
\end{equation}
\begin{equation}
S_{21} = S_{34} = G\cdot \exp(j\pi-j\omega\tau)
\end{equation}
\begin{equation}
S_{24} = S_{31} = G\cdot \exp(-j\omega\tau)
\end{equation}

The scattering matrix of the voltage controlled voltage source
depicted in fig. \ref{fig:xCVS} (left) writes as follows ($\tau$ is
time delay).

\begin{equation}
S_{11} = S_{23} = S_{32} = S_{44} = 1
\end{equation}
\begin{equation}
S_{12} = S_{13} = S_{14} = S_{22} = S_{33} = S_{41} = S_{42} = S_{43} = 0
\end{equation}
\begin{equation}
S_{21} = S_{34} = G\cdot \exp(-j\omega\tau)
\end{equation}
\begin{equation}
S_{24} = S_{31} = G\cdot \exp(j\pi-j\omega\tau)
\end{equation}

\begin{figure}[ht]
\begin{center}
\includegraphics[width=8cm]{xCVS}
\end{center}
\caption{voltage controlled voltage source (left) and current controlled voltage source (right)}
\label{fig:xCVS}
\end{figure}
\FloatBarrier

The scattering matrix of the current controlled voltage source
depicted in fig. \ref{fig:xCVS} (right) writes as follows ($\tau$ is
time delay).

\begin{equation}
S_{14} = S_{23} = S_{32} = S_{41} = 1
\end{equation}
\begin{equation}
S_{11} = S_{12} = S_{13} = S_{22} = S_{33} = S_{42} = S_{43} = S_{44} = 0
\end{equation}
\begin{equation}
S_{21} = S_{34} = \frac{G}{2}\cdot \exp(-j\omega\tau)
\end{equation}
\begin{equation}
S_{24} = S_{31} = \frac{G}{2}\cdot \exp(j\pi-j\omega\tau)
\end{equation}

\subsection*{Transmission Line}
\addcontentsline{toc}{subsection}{Transmission Line}

The scattering matrix of an ideal, lossless transmission line with
impedance $Z$ and electrical length $L$ writes as follows.

\begin{equation}
r = \frac{Z-Z_0}{Z+Z_0}
\end{equation}
\begin{equation}
p = \exp(-j\omega\frac{L}{c})
\end{equation}
\begin{equation}
S_{11} = S_{22} = \frac{r\cdot(1-p^2)}{1-r^2\cdot p^2} \qquad,\qquad
S_{12} = S_{21} = \frac{p\cdot(1-r^2)}{1-r^2\cdot p^2}
\end{equation}

With $c$ = 299 792 458 m/s being the vacuum light velocity.


\chapter*{Noise Waves}
\addcontentsline{toc}{chapter}{Noise Waves}

\section*{Definition}
\addcontentsline{toc}{section}{Definition}

In microwave circuits described by scattering parameters, it is
advantageous to regard noise as noise waves.  The noise
characteristics of an n-port is then defined completely by one
outgoing noise wave $\underline{b}_{noise,n}$ at each port (see 2-port
example in fig. \ref{fig:Sparam}) and the correlation between these
noise sources.  Therefore, mathematically, you can characterize a
noisy n-port by its $n\times n$ scattering matrix $(\underline{S})$
and its $n\times n$ noise wave correlation matrix $(\underline{C})$.

\begin{equation}
\begin{split}
(\underline{C}) =
\begin{pmatrix}
 \overline{\underline{b}_{noise,1}\cdot \underline{b}_{noise,1}^*} &
    \overline{\underline{b}_{noise,1}\cdot \underline{b}_{noise,2}^*} &
    \ldots & \overline{\underline{b}_{noise,1}\cdot \underline{b}_{noise,n}^*}\\
 \overline{\underline{b}_{noise,2}\cdot \underline{b}_{noise,1}^*} &
    \overline{\underline{b}_{noise,2}\cdot \underline{b}_{noise,2}^*} & \ldots &
    \overline{\underline{b}_{noise,2}\cdot \underline{b}_{noise,n}^*}\\
 \vdots & \vdots & \ddots & \vdots\\
 \overline{\underline{b}_{noise,n}\cdot \underline{b}_{noise,1}^*} &
    \overline{\underline{b}_{noise,n}\cdot \underline{b}_{noise,2}^*} &
    \ldots & \overline{\underline{b}_{noise,n}\cdot \underline{b}_{noise,n}^*}\\
\end{pmatrix}
\\ =
\begin{pmatrix}
  \underline{c}_{11} & \underline{c}_{12} & \ldots & \underline{c}_{1n}\\
  \underline{c}_{21} & \underline{c}_{22} & \ldots & \underline{c}_{2n}\\
  \vdots & \vdots & \ddots & \vdots\\
  \underline{c}_{n1} & \underline{c}_{n2} & \ldots & \underline{c}_{nn}\\
\end{pmatrix}
\end{split}
\end{equation}

Where $\overline{x}$ is the time average of $x$ and $\underline{x}^*$
is the conjugate complex of $\underline{x}$.  As can be seen, the
following equations hold:

\begin{equation}
\text{Im}(\underline{c}_{nn}) = \text{Im}(\overline{|b_{noise,n}|^2}) = 0
\end{equation}
\begin{equation}
\underline{c}_{nm} = \underline{c}_{mn}^*
\end{equation}

Where $\text{Im}(\underline{x})$ is the imaginary part of
$\underline{x}$ and $|\underline{x}|$ is the magnitude of
$\underline{x}$.

\begin{figure}[ht]
\begin{center}
\includegraphics[width=8cm]{Sparam}
\end{center}
\caption{Signal flow graph of a noise 2-port}
\label{fig:Sparam}
\end{figure}
\FloatBarrier


\section*{Noise Parameters}
\addcontentsline{toc}{section}{Noise Parameters}

Having the noise wave correlation matrix, one can easily compute the
noise parameters.  The following equations calculate them with regard
to port 1 (input) and port 2 (output).
\\ \\Noise figure:
\begin{equation}
F = 1 + \frac{\underline{c}_{22}}{k\cdot T_0\cdot |\underline{S}_{21}|^2}
\end{equation}
\begin{equation}
NF\,[\text{dB}]\, = 10\cdot\lg F
\end{equation}
\\ \\Optimal source reflection coefficient:
\begin{equation}
\Gamma_{opt} = \frac{\eta_2}{2}\cdot\left( 1-\sqrt{1-\frac{4}{|\eta_2|^2}} \right)
\end{equation}
With
\begin{equation}
\eta_1 = \underline{c}_{11}\cdot |\underline{S}_{21}|^2
       - \text{Re}(\underline{c}_{12}\cdot\underline{S}_{21}\cdot\underline{S}_{11}^*)
       + \underline{c}_{22}\cdot|\underline{S}_{11}|^2
\end{equation}
\begin{equation}
\eta_2 = \frac{\underline{c}_{22} + \eta_1}
              {\underline{c}_{22}\cdot\underline{S}_{11} - \underline{c}_{12}\cdot\underline{S}_{21}}
\end{equation}
\\ \\Minimum noise figure:
\begin{equation}
F_{min} = 1 + \frac{\underline{c}_{22} - \eta_1\cdot |\Gamma_{opt}|^2}
                   {k\cdot T_0\cdot |\underline{S}_{21}|^2\cdot (1+|\Gamma_{opt}|^2)}
\end{equation}
\begin{equation}
NF_{min} = 10\cdot \lg F_{min}
\end{equation}
\\ \\Equivalent noise resistance:
\begin{equation}
R_n = \frac{Z_0}{4\cdot k\cdot T_0}\cdot
      \left( \underline{c}_{11} - 2\cdot
      \text{Re}\left( \underline{c}_{12}\cdot\left( \frac{1+\underline{S}_{11}}{\underline{S}_{21}} \right)^*
      \right)\, + \underline{c}_{22}\cdot
      \left| \frac{1+\underline{S}_{11}}{\underline{S}_{21}} \right|^2 \right)
\end{equation}
\begin{tabular}{ll}
With & \quad Boltzmann constant $k = 1.380658\cdot 10^{-23}$ J/K\\
     & \quad standard temperature $T_0 = 290$K\\
\end{tabular}


\section*{Noise Wave Correlation Matrix in CAE}
\addcontentsline{toc}{section}{Noise Wave Correlation Matrix in CAE}

We have the noise wave correlation ( $(\underline{C})$,
$(\underline{D})$ ) and the scattering matrix ( $(\underline{S})$,
$(\underline{T})$ ) of two arbitrary circuits and want to know the
correlation matrix of the special circuit, that results from
connecting port $m$ of circuit 1 with port $k$ of circuit 2
(fig. \ref{fig:nconnect}).  The following equations perform this
operation:

\begin{equation}
\underline{c}_{nn}' = \underline{c}_{nn} + \left( \underline{d}_{kk} +
   \underline{c}_{mm}\cdot|\underline{T}_{kk}|^2 \right) \cdot
   \left| \frac{\underline{S}_{nm}}{1-\underline{S}_{mm}\cdot\underline{T}_{kk}} \right|^2
   + 2\cdot\text{Re}\left( \underline{c}_{nm}\cdot
   \frac{\underline{T}_{kk}\cdot\underline{S}_{nm}}{1-\underline{S}_{mm}\cdot\underline{T}_{kk}} \right)
\end{equation}
\begin{equation}
\begin{split}
\underline{c}_{nl}' = (\underline{c}_{ln}')^* = 
   (\underline{c}_{mm}\cdot\underline{T}_{kk} + \underline{d}_{kk}\cdot\underline{S}_{mm}^*)\cdot
   \frac{\underline{S}_{nm}\cdot\underline{T}_{lk}^*}{|1-\underline{S}_{mm}\cdot\underline{T}_{kk}|^2}
\\ + \;\underline{c}_{nm}\cdot
     \left(\frac{\underline{T}_{lk}}{1-\underline{S}_{mm}\cdot\underline{T}_{kk}}\right)^*
   + \underline{d}_{kl}\cdot
     \frac{\underline{S}_{nm}}{1-\underline{S}_{mm}\cdot\underline{T}_{kk}}
\end{split}
\end{equation}

\begin{figure}[ht]
\begin{center}
\includegraphics[width=8cm]{nconnect}
\end{center}
\caption{Connecting two noisy circuits}
\label{fig:nconnect}
\end{figure}
\FloatBarrier


\section*{Noise Wave Correlation Matrix of Components}
\addcontentsline{toc}{section}{Noise Wave Correlation Matrix of Components}

Many components do not produce any noise. Every element of their noise
correlation matrix therefore equals exactly zero. Examples are
capacitors, inductors, transformers, voltage and current sources.

\subsection*{Resistor}
\addcontentsline{toc}{subsection}{Resistor}

Being on temperature $T$, the noise wave correlation matrix of an
ideal, ohmic resistor with resistance $R$ writes as follows.

\begin{equation}
(\underline{C}) = k\cdot T\cdot\frac{4\cdot R\cdot Z_0}{(2\cdot Z_0+R)^2}\cdot
\begin{pmatrix}
   1 & -1\\
  -1 &  1\\
\end{pmatrix}
\end{equation}

\chapter*{DC Analysis}
\addcontentsline{toc}{chapter}{DC Analysis}

\section*{Extensions to the MNA}
\addcontentsline{toc}{section}{Extensions to the MNA}

\subsection*{Voltage controlled current source}
\addcontentsline{toc}{subsection}{Voltage controlled current source}

The voltage-dependent current source (VCCS), as shown in fig.
\ref{fig:vccs}, is designated by the following equation which
introduces one more unknown in the MNA matrix.

\begin{figure}[ht]
\begin{center}
\includegraphics[width=4cm]{vccs}
\end{center}
\caption{voltage controlled current source}
\label{fig:vccs}
\end{figure}
\FloatBarrier

\begin{equation}
I_{out} = G\cdot\left(V_{1} - V_{2}\right)
\quad \rightarrow \quad
V_{1} - V_{4} - \frac{1}{G}\cdot I_{out} = 0
\label{eq:vccs}
\end{equation}

The new unknown variable $I_{out}$ must be considered by the four
remaining simple equations.

\begin{equation}
I_{1} = 0 \quad I_{2} = I_{out} \quad I_{3} = -I_{out} \quad I_{4} = 0
\end{equation}

And in matrix representation this is:
\begin{equation}
\begin{pmatrix}
.&.&.&.& 0\\
.&.&.&.& 1\\
.&.&.&.& -1\\
.&.&.&.& 0\\
1 & 0 & 0 & -1 & -\frac{1}{G}
\end{pmatrix}
\cdot
\begin{pmatrix}
V_{1}\\
V_{2}\\
V_{3}\\
V_{4}\\
I_{out}\\
\end{pmatrix}
=
\begin{pmatrix}
I_{1}\\
I_{2}\\
I_{3}\\
I_{4}\\
0\\
\end{pmatrix}
\end{equation}

As you can see the last row which has been added by the VCCS
represents the designating equation (\ref{eq:vccs}).  The additional
right hand column in the matrix keeps the system consistent.

\subsection*{Voltage controlled voltage source}
\addcontentsline{toc}{subsection}{Voltage controlled voltage source}

The voltage-dependent voltage source (VCVS), as shown in fig.
\ref{fig:vcvs}, is designated by the following equation which
introduces one more unknown in the MNA matrix.

\begin{figure}[ht]
\begin{center}
\includegraphics[width=4cm]{vcvs}
\end{center}
\caption{voltage controlled voltage source}
\label{fig:vcvs}
\end{figure}
\FloatBarrier

\begin{equation}
V_{2} - V_{3} = G\cdot \left(V_{1} - V_{4}\right)
\quad \rightarrow \quad
V_{1}\cdot G - V_{2} + V_{3} - V_{4}\cdot G = 0
\label{eq:vcvs}
\end{equation}

The new unknown variable $I_{out}$ must be considered by the four
remaining simple equations.

\begin{equation}
I_{1} = 0 \quad I_{2} = -I_{out} \quad I_{3} = I_{out} \quad I_{4} = 0
\end{equation}

And in matrix representation this is:
\begin{equation}
\begin{pmatrix}
.&.&.&.& 0\\
.&.&.&.& -1\\
.&.&.&.& 1\\
.&.&.&.& 0\\
G & -1 & 1 & -G & 0
\end{pmatrix}
\cdot
\begin{pmatrix}
V_{1}\\
V_{2}\\
V_{3}\\
V_{4}\\
I_{out}\\
\end{pmatrix}
=
\begin{pmatrix}
I_{1}\\
I_{2}\\
I_{3}\\
I_{4}\\
0\\
\end{pmatrix}
\end{equation}

\subsection*{Current controlled current source}
\addcontentsline{toc}{subsection}{Current controlled current source}

The current-dependent current source (CCCS), as shown in fig.
\ref{fig:cccs}, is designated by the following equation which
introduces one more unknown in the MNA matrix.

\begin{figure}[ht]
\begin{center}
\includegraphics[width=4cm]{cccs}
\end{center}
\caption{current controlled current source}
\label{fig:cccs}
\end{figure}
\FloatBarrier

\begin{equation}
V_{1} - V_{4} = 0
\label{eq:cccs}
\end{equation}

The new unknown variable $I_{out}$ must be considered by the four
remaining simple equations.

\begin{equation}
I_{1} = -\frac{1}{G}\cdot I_{out} \quad I_{2} = I_{out} \quad I_{3} = -I_{out} \quad I_{4} = \frac{1}{G}\cdot I_{out}
\end{equation}

And in matrix representation this is:
\begin{equation}
\begin{pmatrix}
.&.&.&.& -\frac{1}{G}\\
.&.&.&.& 1\\
.&.&.&.& -1\\
.&.&.&.& \frac{1}{G}\\
1 & 0 & 0 & -1 & 0
\end{pmatrix}
\cdot
\begin{pmatrix}
V_{1}\\
V_{2}\\
V_{3}\\
V_{4}\\
I_{out}\\
\end{pmatrix}
=
\begin{pmatrix}
I_{1}\\
I_{2}\\
I_{3}\\
I_{4}\\
0\\
\end{pmatrix}
\end{equation}

\subsection*{Current controlled voltage source}
\addcontentsline{toc}{subsection}{Current controlled voltage source}

The current-dependent voltage source (CCVS), as shown in fig.
\ref{fig:ccvs}, is designated by the following equations which
introduces two more unknowns in the MNA matrix.

\begin{figure}[ht]
\begin{center}
\includegraphics[width=4cm]{ccvs}
\end{center}
\caption{current controlled voltage source}
\label{fig:ccvs}
\end{figure}
\FloatBarrier

\begin{equation}
V_{1} - V_{4} = 0
\end{equation}
\begin{equation}
V_{2} - V_{3} = G\cdot I_{in}
\quad \rightarrow \quad
V_{2} - V_{3} - I_{in}\cdot G = 0
\label{eq:ccvs}
\end{equation}

The new unknown variables $I_{out}$ and $I_{in}$ must be considered by
the four remaining simple equations.

\begin{equation}
I_{1} = -I_{in} \quad I_{2} = -I_{out} \quad I_{3} = I_{out} \quad I_{4} = I_{in}
\end{equation}

The matrix representation needs to be augmented by two more new rows
(for the new unknown variables) and their corresponding columns.
\begin{equation}
\begin{pmatrix}
.&.&.&.& -1 & 0\\
.&.&.&.& 0 & -1\\
.&.&.&.& 0 & 1\\
.&.&.&.& 1 & 0\\
0 & 1 & -1 & 0 & -G & 0\\
1 & 0 & 0 & -1 & 0 & 0
\end{pmatrix}
\cdot
\begin{pmatrix}
V_{1}\\
V_{2}\\
V_{3}\\
V_{4}\\
I_{in}\\
I_{out}
\end{pmatrix}
=
\begin{pmatrix}
I_{1}\\
I_{2}\\
I_{3}\\
I_{4}\\
0\\
0
\end{pmatrix}
\end{equation}

\subsection*{Operational amplifier}
\addcontentsline{toc}{subsection}{Operational amplifier}

\subsection*{Transformator}
\addcontentsline{toc}{subsection}{Transformator}

\subsection*{Gyrator}
\addcontentsline{toc}{subsection}{Gyrator}

\nocite{Compton}
\nocite{Queiroz}
\nocite{Wedepohl}
\nocite{Monaco}

\chapter*{Bibliography}
\addcontentsline{toc}{chapter}{Bibliography}
\def\chapter*{}% to suppress any header
\def\section*{}% just to be sure
\renewcommand{\bibname}{}
\bibliographystyle{plaindin}
\bibliography{literature}

\end{document}
