\documentclass[10pt]{report}
\usepackage{a4wide}
\usepackage{epsfig}
\usepackage{array}
\usepackage{amsmath}
\usepackage{SIunits}
\usepackage{psfrag}
\usepackage{relsize}
\usepackage[section]{placeins}
\usepackage{listings}

\newif\ifpdf
\ifx\pdfoutput\undefined
  \pdffalse
\else
  \pdfoutput=1
  \pdftrue
\fi

\ifpdf
\pdfcompresslevel=9
\pdfinfo {
  /Title   (Qucs)
  /Subject (Technical Papers)
  /Author  (Stefan Jahn)
}
\fi

\makeatletter
\def\thickhrulefill{\leavevmode \leaders \hrule height 1pt\hfill \kern \z@}
\renewcommand{\maketitle}{\begin{titlepage}%
    \let\footnotesize\small
    \let\footnoterule\relax
    \parindent \z@
    \reset@font
    \null\vfil
    \vspace*{3cm}
    \begin{flushleft}
      \bf \huge \@title
    \end{flushleft}
    \par
    \hrule height 3pt
    \par
    \begin{flushright}
      \LARGE Technical Papers \par
    \end{flushright}
    \vskip 60\p@
    \vfill


    \begin{flushright}
      \Large \@author \par
    \end{flushright}

    \hrule height 3pt \par

\vspace*{24pt}

Copyright \copyright{} 2003 Michael Margraf 
\textless margraf@mwt.ee.tu-berlin.de\textgreater \par
Copyright \copyright{} 2003 Stefan Jahn 
\textless jahn@mwt.ee.tu-berlin.de\textgreater \par

\vspace*{12pt}

Permission is granted to copy, distribute and/or modify this document
under the terms of the GNU Free Documentation License, Version 1.1 or
any later version published by the Free Software Foundation.  A copy
of the license is included in the section entitled "GNU Free
Documentation License".

\vspace*{1cm}

  \end{titlepage}%
  \setcounter{footnote}{0}%
}
\makeatother

\author{Michael Margraf \\ Stefan Jahn}
\title{Qucs}
\date{}

\begin{document}

\maketitle

\tableofcontents

\setlength{\parindent}{0pt}
\newpage

\chapter*{Mathematical expressions and conversions}
\addcontentsline{toc}{chapter}{Mathematical expressions and conversions}

\section*{Scattering parameters}
\addcontentsline{toc}{section}{Scattering parameters}

\subsection*{Recalculating $\mathbf{50\ohm}$-S-parameters for arbitrary port impedances}
\addcontentsline{toc}{subsection}{Recalculating $50\ohm$-S-parameters for arbitrary port impedances}

During S-parameter simulation it is necessary to have all components
in a circuit normalized to the same impedance.  In the field of high
frequency techniques this is usually $50\ohm$.  In order to allow port
impedances other than $50\ohm$ in a simulation the following two step
process must be applied to the resulting S-parameter analysis.

\begin{equation}
\left[\underline{N}\right] = 
\left(\left[\underline{S}\right] - \left[\underline{R}\right]\right) \cdot
\left(\left[\underline{E}\right] - \left[\underline{R}\right] \cdot \left[\underline{S}\right]\right)^{-1}
\end{equation}

\begin{equation}
\underline{N}_{nm} = \underline{S}_{nm}\cdot \sqrt{\dfrac{Z^{m}}{Z^{n}}}\cdot
\dfrac{Z^{n} + Z_{0}}{Z^{m} + Z_{0}}
\end{equation}

With\\

\begin{tabular}{rll}
$Z_{0}$ & = & $50\ohm$\\& &\\
$\left[\underline{E}\right]$ & = &
$\begin{pmatrix}
1 & 0 & \ldots & 0\\
0 & 1 & \ldots & 0\\
\vdots & \vdots & \ddots & \vdots\\
0 & 0 & \ldots & 1\\
\end{pmatrix}$
identity matrix\\& &\\
$\left[\underline{S}\right]$ & = & original $50\ohm$-S-parameter matrix\\& &\\
$\left[\underline{N}\right]$ & = & recalculated scattering matrix\\& &\\
$\left[\underline{R}\right]$ & = &
$\begin{pmatrix}
\underline{r}(Z_{1}) & 0 & \ldots & 0\\
0 & \underline{r}(Z_{2}) & \ldots & 0\\
\vdots & \vdots & \ddots & \vdots\\
0 & 0 & \ldots & \underline{r}(Z_{n})\\
\end{pmatrix}$
reflection coefficient matrix\\& &\\
$\underline{r}(Z_{n})$ & = &
$\dfrac{Z_{n} - Z_{0}}{Z_{n} + Z_{0}}$
reflection coefficient of impedance at port n\\& &\\
\end{tabular}

And furthermore\\

\begin{tabular}{rll}
$\left[\underline{X}\right]^{-1}$ & = & 
inverted matrix of $\left[\underline{X}\right]$\\& &\\
$\underline{X}_{nm}$ & = & 
element of matrix $\left[\underline{X}\right]$ at row n and column m\\& &\\
\end{tabular}

\subsection*{Differential S-parameter ports}
\addcontentsline{toc}{subsection}{Differential S-parameter ports}

The implemented algorithm for then S-parameter analysis calculates
S-parameters in terms of the ground node.  In order to allow
differential S-parameters as well it is necessary to insert an ideal
impedance transformer with a turns ratio of 1:1 between the
differential port and the device under test.

\begin{figure}[ht]
\begin{center}
\includegraphics[width=12cm]{differential}
\end{center}
\caption{transformation of differential port into single ended port}
\label{fig:differential}
\end{figure}
\FloatBarrier

The S-parameter matrix of the inserted ideal transformer being a three
port device can be written as follows.

\begin{equation}
\begin{pmatrix}
S
\end{pmatrix}
= \dfrac{1}{3}\cdot
\begin{pmatrix}
1 & 2 & -2\\
2 & 1 & 2\\
-2 & 2 & 1\\
\end{pmatrix}
\end{equation}

This transformation can be applied to each S-parameter port in a
circuit regardless whether it is actually differential or not.

\subsection*{Symmetrical transformator}
\addcontentsline{toc}{subsection}{Symmetrical transformator}

\begin{figure}[ht]
\begin{center}
\includegraphics[width=4cm]{symtrafo}
\end{center}
\caption{symmetrical transformator}
\label{fig:symtrafo}
\end{figure}
\FloatBarrier

\end{document}
