%
% This document contains the chapter about coplanar components.
%
% Copyright (C) 2004 Vincent Habchi, F5RCS <10.50@free.fr>
% Copyright (C) 2004 Stefan Jahn <stefan@lkcc.org>
%
% Permission is granted to copy, distribute and/or modify this document
% under the terms of the GNU Free Documentation License, Version 1.1
% or any later version published by the Free Software Foundation.
%

\chapter{Coplanar components}
%\addcontentsline{toc}{chapter}{Coplanar components}

\section{Coplanar waveguides (CPW)}

\subsection{Definition}

A \emph{coplanar line} is a structure in which all the conductors
supporting wave propagation are located on the same plane,
i.e. generally the top of a dielectric substrate. There exist two main
type of coplanar lines: the first, called \emph{coplanar waveguide
(CPW)}, that we will study here, is composed of a median metallic
strip separated by two narrow slits from a infinite ground plane, as
may be seen on the figure below.

\begin{figure}[ht]
\begin{center}
\includegraphics[width=0.9\linewidth]{cpline}
\end{center}
\caption{coplanar waveguide line}
\label{fig:CoplanarLine}
\end{figure}
\FloatBarrier

The characteristic dimensions of a CPW are the central strip width $W$
and the width of the slots $s$. The structure is obviously symmetrical
along a vertical plane running in the middle of the central strip.

\begin{figure}[ht]
\begin{center}
\includegraphics[width=0.5\linewidth]{cpscheme}
\end{center}
\label{fig:CoplanarScheme}
\end{figure}
\FloatBarrier

The other coplanar line, called a \emph{coplanar slot (CPS)} is the
complementary of that topology, consisting of two strips running side
by side.

\subsection{Quasi-static analysis by conformal mappings}

A CPW can be quasi-statically analysed through the use of
\emph{conformal mappings}. Briefly speaking, it consists in
transforming the geometry of the PCB into another conformation, whose
properties make the computations straightforward. The interested
reader can consult the pp. 886~-~910 of \cite{Collin} which has a
correct coverage of both the theoretical and applied methods. The
French reader interested in the mathematical arcanes involved is
referred to the second chapter of \cite{Mir} (which may be out of
print nowadays), for an extensive review of all the theoretical
framework. The following analysis is mainly borrowed from
\cite{Gupta1}, pp.~375 \emph{et seq}. with addition from
\cite{Collin}.

\addvspace{12pt}

The CPW of negligible thickness located on top of an infinitely deep
substrate, as shown on the left of the figure below, can be mapped
into a parallel plate capacitor filled with dielectric $ABCD$ using
the conformal function:
\begin{equation}
w = \int_{z_0}^{z}�\dfrac{dz}{\sqrt{(z-W/2)(z-W/2-s)}}.
\end{equation}

\begin{figure}[ht]
\begin{center}
\includegraphics[width=0.7\linewidth]{cpconform}
\end{center}
\label{fig:CoplanarConformal}
\end{figure}
\FloatBarrier

To further simplify the analysis, the original dielectric boundary is
assumed to constitute a magnetic wall, so that $BC$ and $AD$ becomes
magnetic walls too and there is no resulting fringing field in the
resulting capacitor. With that assumption, the capacitance per unit
length is merely the sum of the top (air filled) and bottom
(dielectric filled) partial capacitances. The latter is given by:
\begin{equation}
C_d = 2\cdot \varepsilon_0\cdot \varepsilon_r\cdot \dfrac{K(k_1)}{K'(k_1)}
\end{equation}

while the former is:
\begin{equation}
C_a = 2\cdot \varepsilon_0\cdot \dfrac{K(k_1)}{K'(k_1)}
\end{equation}

In both formulae $K(k)$ and $K'(k)$ represent the complete elliptic
integral of the first kind and its complement, and $k_1=
\tfrac{W}{W+2s}$.  While the separate evaluation of $K$ and $K'$ is
more or less tricky, the $K/K'$ ratio lets itself compute efficiently
through the following formulae:
\begin{equation}
\dfrac{K(k)}{K'(k)} =
\dfrac{\pi}{\ln\left(2\tfrac{1+\sqrt{k'}}{1-\sqrt{k'}}\right)}
\;\;\;\; \textrm{ for } \;\;\;\;
0 \le k \le \dfrac{1}{\sqrt{2}}
\end{equation}
\begin{equation}
\dfrac{K(k)}{K'(k)} =
\dfrac{\ln\left(2\tfrac{1+\sqrt{k}}{1-\sqrt{k}}\right)}{\pi}
\;\;\;\; \textrm{ for } \;\;\;\;
\dfrac{1} {\sqrt{2}} \le k \le 1
\end{equation}

with $k'$ being the complementary modulus: $k'=\sqrt{1-k^2}$.  While
\cite{Collin} states that the accuracy of the above formulae is close
to $10^{-5}$, \cite{Gupta1} claims it to be $3.10^{-6}$.  It can be
considered as exact for any practical purposes.

\addvspace{12pt}

The total line capacitance is thus the sum of $C_d$ and $C_a$.  The
effective permittivity is therefore:
\begin{equation}
\varepsilon_{re} = \dfrac{\varepsilon_r+1}{2}
\end{equation}

and the impedance:
\begin{equation}
Z=\dfrac{30\pi}{\sqrt{\varepsilon_{re}}}\cdot\dfrac{K'(k_1)}{K(k_1)}
\label{eq:CoplanarZL}
\end{equation}

\begin{figure}[ht]
\begin{center}
\psfrag{impedance ZL in Ohm}{$\mathrm{\text{impedance }Z_{L}\text{ in }[\ohm]}$}
\psfrag{normalised strip width W/s}{normalised strip width W/s}
\includegraphics[width=0.95\linewidth]{coplanarzl}
\end{center}
\caption{characteristic impedance as approximated by eq. \eqref{eq:CoplanarZL} for $\varepsilon_{r}$ = $1.0$ (air), $3.78$ (quartz) and $9.5$ (alumina)}
\label{fig:coplanarzl}
\end{figure}
\FloatBarrier

In practical cases, the substrate has a finite thickness $h$. To carry
out the analysis of this conformation, a preliminary conformal mapping
transforms the finite thickness dielectric into an infinite thickness
one. Only the effective permittivity is altered; it becomes:
\begin{equation}
\varepsilon_{re}=1+\dfrac{\varepsilon_r-1}{2}\cdot\dfrac{K(k_2)}{K'(k_2)}\cdot\dfrac{K'(k_1)}{K(k_1)}
\end{equation}

where $k_1$ is given above and
\begin{equation}
k_2=\dfrac{\sinh\left(\dfrac{\pi W}{4h}\right)}{\sinh\left(\dfrac{\pi\cdot\left(W+2s\right)}{4h}\right)}.
\end{equation}

Finally, let us consider a CPW over a finite thickness dielectric
backed by an infinite ground plane. In this case, the quasi-TEM wave
is an hybrid between microstrip and true CPW mode. The equations then
become:
\begin{equation}
\varepsilon_{re}=1+q\cdot\left(\varepsilon_r -1\right)
\end{equation}

where $q$, called \emph{filling factor} is given by:
\begin{equation}
q = \dfrac{\dfrac{K(k_3)}{K'(k_3)}}{\dfrac{K(k_1)}{K'(k_1)}+\dfrac{K(k_3)}{K'(k_3)}}
\end{equation}

and
\begin{equation}
k_3 = \dfrac{\tanh\left(\dfrac{\pi W}{4h}\right)}{\tanh\left(\dfrac{\pi\cdot\left(W+2s\right)}{4h}\right)}
\end{equation}

The impedance of this line amounts to:
\begin{equation}
Z=\dfrac{60\pi}{\sqrt{\varepsilon_{re}}}\cdot\dfrac{1}{\dfrac{K(k_1)}{K'(k_1)}+\dfrac{K(k_3)}{K'(k_3)}}
\end{equation}

\subsection{Effects of metalization thickness}

In most practical cases, the strips are very thin, yet their thickness
cannot be entirely neglected. A first order correction to take into
account the non-zero thickness of the conductor consists in
\cite{Gupta1}:
\begin{equation}
s_e = s - \Delta
\end{equation}

and
\begin{equation}
W_e = W + \Delta
\end{equation}

where
\begin{equation}
\Delta = \dfrac{1.25t}{\pi}\cdot\left(1 + \ln\left(\dfrac{4 \pi W}{t}\right)\right)
\end{equation}

In the computation of the impedance, both the $k_1$ and the effective
dielectric constant are affected, wherefore $k_1$ must be substituted
by an ``effective'' modulus $k_e$, with:
\begin{equation}
k_e = \dfrac {W_e}{W_e + 2 s_e} \approx k_1 + \left(1 - k_1^2\right)\cdot \dfrac{\Delta}{2s}
\end{equation}

and
\begin{equation}
\varepsilon_{re}^{t} = \varepsilon_{re} - \dfrac{0.7\cdot\left(\varepsilon_{re} - 1\right)\cdot\dfrac{t}{s}}{\dfrac{K(k_1)}{K'(k_1)} + 0.7\cdot \dfrac{t}{s}}
\end{equation}

\subsection{Effects of dispersion}

The effects of dispersion in CPW are similar to those encountered in
the microstrip lines, though the net effect on impedance is somewhat
different. \cite{Gupta1} gives a closed form expression to compute
$\varepsilon_{re}(f)$ from its quasi-static value:
\begin{equation}
\sqrt{\varepsilon_{re}(f)}=\sqrt{\varepsilon_{re}(0)} + \dfrac{\sqrt{\varepsilon_r} - \sqrt{\varepsilon_{re}(0)}}{1+G\cdot\left(\dfrac{f}{f_{TE}}\right)^{-1.8}}
\end{equation}

where:
\begin{align}
G &= e^{u\cdot \ln\left(\tfrac{W}{s}\right)+v}\\
u &= 0.54 - 0.64p + 0.015p^2\\
v &= 0.43-0.86p+0.54p^2\\
p &= \ln\left(\tfrac{W}{h}\right)
\end{align}

$f_{TE}$ is the cutoff frequency of the ${TE}_{0}$ mode, equal to:
\begin{equation}
f_{TE} = \dfrac{c}{4h\cdot\sqrt{\varepsilon_r - 1}}.
\end{equation}

The accuracy of this expression is claimed to be better than 5\% for
$0.1 \le W/h \le 5$, $0.1 \le s/W
\le 5$, $1.5 \le \varepsilon_r \le 50$ and $0 \le f/f_{TE} \le 10$.

\subsection{Evaluation of losses}

As for microstrip lines, the losses in CPW results of at least two
factors: a dielectric loss $\alpha_d$ and conductor losses
$\alpha_c^{CW}$.  The dielectric loss $\alpha_d$ is identical to the
microstrip case.
