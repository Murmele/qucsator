%
% This document contains an introduction to the grammars and some tex.
%
% Copyright (C) 2005 Raimund Jacob <raimi@lkcc.org>
%
% Permission is granted to copy, distribute and/or modify this document
% under the terms of the GNU Free Documentation License, Version 1.1
% or any later version published by the Free Software Foundation.
%

\chapter{Qucs file formats}
\label{sec:qucsfiles}

%
% Grammar environment based on Martin Grabmueller's original one
%
\newcommand\tok[1]{`{\bf #1}'}  % Literal token
\newcommand\ntok[1]{{\bf #1}}   % Composed token
\newcommand\bsl[1]{\tok{$\backslash$#1}}
\newenvironment{grammar}%
  {\newcommand\produces[2]{##1 \> $\rightarrow$ \> ##2 \\}
   \newcommand\orproduces[1]{\> \> \makebox[0pt][r]{$|$ }##1 \\}
   \newcommand\opt[1]{[ ##1 ]}
   \newcommand\rep[1]{\{ ##1 \}}
   \newcommand\alt[0]{$|$}
   \newcommand\group[1]{( ##1 )}
   \newcommand\emptyprod{{$\varepsilon$}}
   \newcommand\heading[1]{\rule{\linewidth}{1pt} \\{\bf ##1}\\[2ex]}
   \newcommand\separator{\rule{\linewidth}{1pt} \\}
   \begin{tabbing}
   \qquad\qquad\qquad\qquad \= \qquad \= \kill}
  {\end{tabbing}}

\def\REalpha{\tok{a} \alt{}~\dots~\alt{} \tok{z} \alt{}
             \tok{A} \alt{}~\dots~\alt{} \tok{Z}}
\def\REalnum{\REalpha \alt{} \tok{0}~\dots~\tok{9} \alt{} \tok{\_}}
\def\REplusminus{\opt{\tok{+} \alt{} \tok{-}}}

%
% How the grammar works
%
Qucs uses plain-text (ASCII) files as its input and tranfer format for
netlists and data. This appendix explains the file formats by
describing the grammars of the languages used. The files are generally
line-oriented but arbitrary whitespace between the token is
allowed. You can also use the backslash ($\backslash$) to continue a
line on the next line. This works almost everywhere but in
comment lines.

The grammars are presented using a version of the Extended Backus-Naur
Form (EBNF) which works as follows:

\vskip0.5em
\begin{tabular}{ll}
$A \rightarrow B$&
Nonterminal $A$ produces sentential form $B$.\\

$B | C$&
Produces $B$ or $C$.\\

$\{ A \}$&
Arbitrary repetition of form $A$. No repetition is allowed as well
(``Kleene operator'').\\

[ $A$ ]&
Form $A$ is optional.\\

$(A)$&
Grouping, stands for $A$ itself.\\

%Not needed (yet): $\varepsilon$& The empty form.\\
\end{tabular}
\vskip0.5em

Nonterminal symbols are set in normal font, terminal symbols are in
bold font. Terminal symbols in single quotes are literally found in
the input while the other terminal symbols are compositions. See below
for definition of composed terminal symbols.

